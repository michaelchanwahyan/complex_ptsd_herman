\documentclass[12pt]{article}
\usepackage{fullpage,graphicx,psfrag,amsmath,amsfonts,verbatim}
\usepackage[small,bf]{caption}

\usepackage{tabularx}

% ---------------------------
% Chinese Characters Packages
% ---------------------------
\usepackage{fontspec}
\usepackage{xeCJK}
\setmainfont{Times}
\setCJKmainfont{BiauKai}

\bibliographystyle{alpha}

\title{Complex PTSD: A Syndrome in Survivors of Prolonged and Repeated Trauma}
\author{Judith Lewis Herman}

\date{}

\begin{document}
\maketitle

\begin{abstract}
    This paper reviews the evidence for the existence of a complex form of
    post-traumatic disorder in survivors of prolonged, repeated trauma. This
    syndrome is currently under consideration for inclusion in DSM-IV under the
    name of DESNOS (Disorders of Extreme Stress Not Otherwise Specified). The
    current diagnostic formulation of PTSD derives primarily from observations
    of survivors of relatively circumscribed traumatic events. This formulation
    fails to capture the protean sequelae of prolonged, repeated trauma. In
    contrast to a single traumatic event, prolonged, repeated trauma can occur
    only where the victim is in a state of captivity, under the control of the
    perpetrator. The psychological impact of subordination to coercive control
    has many common features, whether it occurs within the public sphere of
    politics or within the private sphere of sexual and domestic relations.

    本文回顧了長期、反覆創傷的倖存者存在復雜形式的創傷後障礙的證據。 該綜合徵
    目前正在考慮以 DESNOS(未另行指定的極端應激障礙)的名稱納入 DSM-IV。 當前的
    PTSD 診斷公式主要來自對相對有限的創傷事件倖存者的觀察。 這種表述未能捕捉到
    長期、反覆創傷的千變萬化的後遺症。 與單一的創傷事件相反,只有當受害者處於
    囚禁狀態,在犯罪者的控制下,才會發生長期的、重複的創傷。 從屬於強制控制的
    心理影響具有許多共同特徵,無論它發生在政治的公共領域還是發生在性和
    家庭關係的私人領域。
\end{abstract}

% \newpage
\tableofcontents
\newpage

\section{Introduction 導論}
    The current diagnostic formulation of PTSD derives primarily from
    observations of survivors of relatively circumscribed traumatic events:
    combat, disaster, and rape. It has been suggested that this formulation
    fails to capture the protean sequelae of prolonged, repeated trauma. In
    contrast to the circumscribed traumatic event, prolonged, repeated trauma
    can occur only where the victim is in a state of captivity, unable to flee,
    and under the control of the perpetrator. Examples of such conditions
    include prisons, concentration camps, and slave labor camps. Such
    conditions also exist in some religious cults, in brothels and other
    institutions of organized sexual exploitation, and in some families.

    當前的 PTSD 診斷公式主要來自對相對有限的創傷事件倖存者的觀察:戰鬥、災難和
    強姦。有人認為,這種表述未能捕捉到長期、反覆創傷的千變萬化的後遺症。
    與局限性創傷事件相反,只有當受害者處於囚禁狀態、無法逃脫並處於犯罪者的
    控制之下時,才會發生長期的、反覆的創傷。 此類條件的例子包括監獄、集中營和
    奴隸勞改營。 這種情況也存在於一些宗教崇拜、妓院和其他有組織的性剝削機構以及
    一些家庭中。

    Captivity, which brings the victim into prolunged contact with the
    perpetrator, creates a special type of relationship, one of coercive
    control. This is equally true whether the victim is rendered captive
    primarily by physical force (as in the case of prisoners and hostages), or
    by a combination of physical, economic, social, and psychological means (as
    in the case of religious cult members, battered women, and abused
    children). The psychological impact of subordination to coercive control
    may have many common features, whether that subordination occurs within the
    public sphere of politics or within the supposedly private (but equally
    political) sphere of sexual and domestic relations.

    囚禁使受害者與肇事者長期接觸,創造了一種特殊類型的關係,一種強制控制。
    無論受害者主要是被肉體力量俘虜(如囚犯和人質的情況),還是被身體、經濟、
    社會和心理手段的結合(如宗教信仰的情況),這同樣適用 邪教成員、受虐婦女
    和受虐兒童)。 從屬於強制控制的心理影響可能有許多共同特徵,無論這種從屬
    發生在政治的公共領域,還是發生在假定的私人(但同樣是政治的)性和
    家庭關係領域。

    This paper reviews the evidence for the existence of a complex form of
    post-traumatic disorder in survivors of prolonged, repeated trauma. A
    preliminary formulation of this complex post-traumatic syndrome is
    currently under consideration for inclusion in DSM-IV under the name of
    DESNOS (Disorders of Extreme Stress). In the course of a larger work in
    progress, I have recently scanned literature of the past 50 years on
    suivivors of prolonged domestic, sexual, or political victimization
    (Herman, 1992). This literature includes first-person accounts of survivors
    themselves, descriptive clinical literature, and, where available, more
    rigorously designed clinical studies. In the literature review, particular
    attention was directed toward observations that did not fit readily into
    the existing criteria for PTSD, Though the sources include works by authors
    of many nationalities, only works originally written in English or
    available in English translation were reviewed.

    本文回顧了長期、反覆創傷的倖存者存在復雜形式的創傷後障礙的證據。 這種複雜的
    創傷後綜合症的初步表述目前正在考慮以 DESNOS(極度應激障礙)的名稱納入
    DSM-IV。 在正在進行的一項更大的工作中,我最近瀏覽了過去 50 年關於長期遭受
    家庭、性或政治傷害的倖存者的文獻(Herman,1992)。 這些文獻包括對倖存者
    自身的第一人稱描述、描述性臨床文獻,以及在可能的情況下設計更嚴格的
    臨床研究。在文獻綜述中,特別關注那些不容易符合 PTSD 現有標準的觀察結果,
    儘管來源包括許多國家作者的作品,但只有英文原著或有英文翻譯的作品才被納入,
    審查。

    The concept of a spectrum of post-traumatic disorders has been suggested
    independently by many major contributors to the field. Kolb, in a letter to
    the editor of the American Joumal of Psychiatry (1989), writes of the
    ``heterogeneity" of PTSD. He observes that ``PTSD is to psychiatry as
    syphilis was to medicine. At one time or another PTSD may appear to mimic
    every personality disorder," and notes further that ``It is those
    threatened over long periods of time who suffer the long-standing severe
    personality disorganization." Niederland, on the basis of his work with
    survivors of the Nazi Holocaust, observes that ``the concept of traumatic
    neurosis does not appear sufficient to cover the multitude and severity of
    clinical manifestations" of the survivor syndrome (in Krystal, 1968,
    p. 314). Tanay, working with the same population, notes that ``the
    psychopathology may be hidden in characterological changes that are
    manifest only in disturbed object relationships and attitudes towards work,
    the world, man and God" (Krystal, 1968, p. 221). Similarly, Kroll and his
    colleagues (1989), on the basis of their work with Southeast Asian
    refugees, suggest the need for an ``expanded concept of PTSD that takes
    into account the observations (of the effects of) severe, prolonged, and/or
    massive psychological and physical traumata." Horowitz (1986) suggests the
    concept of a ``post-traumatic character disorder," and Brown and Fromm
    (1986) speak of ``complicated PTSD."

    該領域的許多主要貢獻者獨立提出了一系列創傷後疾病的概念。 科爾布 (Kolb) 在給
    美國精神病學雜誌 (American Joumal of Psychiatry) 編輯的一封信中 (1989)
    提到了 PTSD 的“異質性”。 他觀察到“PTSD之於精神病學就像梅毒之於醫學。
    PTSD 有時可能會模仿每一種人格障礙,”並進一步指出,“正是那些長期受到
    威脅的人,才會遭受長期嚴重的人格障礙。” Niederland 根據他對納粹大屠殺
    倖存者的研究,觀察到“創傷性神經症的概念似乎不足以涵蓋倖存者綜合症臨床
    表現的多樣性和嚴重性”(在 Krystal ,1968 年,第 314 頁)。 Tanay 對同一人群
    進行研究,指出“心理病理學可能隱藏在性格變化中,這些變化僅在受干擾的客體關係
    和對工作、世界、人和上帝的態度中表現出來”(Krys- tal,1968 年,第 221 頁)。
    同樣,Kroll 和他的同事 (1989),基於他們對東南亞難民的工作,建議需要一個
    “擴展的 PTSD 概念,考慮到嚴重、長期(影響)的觀察結果。 ,和/或巨大的心理和
    身體創傷。” Horowitz (1986) 提出了“創傷後性格障礙”的概念,Brown 和 Fromm
    (1986) 談到了“複雜的 PTSD”。

    Clinicians working with survivors of childhood abuse also invoke the need
    for an expanded diagnostic concept. Gelinas (1983) describes the
    ``disguised presentation" of the survivor of childhood sexual abuse as a
    patient with chronic depression complicated by dissociative symptoms,
    substance abuse, impulsivity, self-mutilation, and suicidality. She
    formulates the underlying psychopathology as a complicated traumatic
    neurosis. Goodwin (1988) conceptualizes the sequelae of prolonged childhood
    abuse as a severe post-traumatic syndrome which includes fugue and other
    dissociative states, ego fragmentation, affective and anxiety disorders,
    reenactment and revictimization, somatization and suicidality.

    與兒童虐待倖存者打交道的臨床醫生也提出需要擴展診斷概念。 Gelinas (1983)
    將童年性虐待倖存者的“偽裝表現”描述為患有慢性抑鬱症並伴有解離症狀、
    藥物濫用、衝動、自殘和自殺傾向的患者。 她將潛在的精神病理學表述為複雜的
    創傷性神經症。Goodwin (1988) 將長期童年虐待的後遺症概念化為嚴重的創傷後
    綜合症,包括神遊和其他解離狀態、自我分裂、情感和焦慮障礙、重演和再次受害、
    軀體化和自殺。

    Clinical observations identify three broad areas of disturbance which
    transcend simple PTSD. The first is symptomatic: the symptom picture in
    survivors of prolonged trauma often appears to be more complex, diffuse,
    and tenacious than in simple PTSD. The second is characterological:
    survivors of prolonged abuse develop characteristic personality changes,
    including deformations of relatedness and identity. The third area involves
    the survivor's vulnerability to repeated harm, both self-inflicted and at
    the hands of others.

    臨床觀察確定了超越簡單 PTSD 的三大障礙領域。 第一個是有症狀的:長期創傷
    倖存者的症狀往往比單純的 PTSD 更複雜、更分散、更頑固。 第二個是性格方面的:
    長期虐待的倖存者會出現性格上的變化,包括關係和身份的變形。 第三個方面涉及
    倖存者易受反覆傷害的脆弱性,包括自己造成的和他人的傷害。

\section{Symptomatic Sequelae of Prolonged Victimization
長期受害的症狀性後遺症}

\paragraph{Multiplicity of Symptoms 多種症狀}
    The pathological environment of prolonged abuse fosters the development of
    a prodigious array of psychiatric symptoms. A history of abuse,
    particularly in childhood, appears to be one of the major factors
    predisposing a person to become a psychiatric patient. While only a
    minority of survivors of chronic childhood abuse become psychiatric
    patients, a large proportion (40-70\%) of adult psychiatric patients are
    survivors of abuse (Briere and Runtz, 1987; Briere and Zaidi, 1989, Bryer
    et aL, 1987, Carmen et aL, 1984; Jacobson and Richardson, 1987).

    長期虐待的病理環境促進了一系列驚人的精神症狀的發展。 虐待史,尤其是
    童年時期的虐待史,似乎是使一個人成為精神病患者的主要因素之一。 雖然只有少數
    長期兒童虐待倖存者成為精神病患者,但很大一部分 (40-70\%) 成年精神病患者是
    虐待倖存者(Briere 和 Runtz,1987 年;Briere 和 Zaidi,1989 年,Bryer 等人,
    1987 年, Carmen 等人,1984 年;Jacobson 和 Richardson,1987 年)。

    Survivors who become patients present with a great number and variety of
    complaints. Their general levels of distress are higher than those of
    patients who do not have abuse histories. Detailed inventories of their
    symptoms reveal significant pathology in multiple domains: somatic,
    cognitive, affective, behavioral, and relational. Bryer and his colleagues
    (1987), studying psychiatric inpatients, report that women with histories
    of physical or sexual abuse have significantly higher scores than other
    patients on standardized measures of somatization, depression, general and
    phobic anxiety, interpersonal sensitivity, paranoia, and ``psychoticism"
    (dissociative symptoms were not measured specifically). Briere (1988),
    studying outpatients at a crisis intervention service, reports that
    survivors of childhood abuse display significantly more insomnia, sexual
    dysfunction, dissociation, anger, suicidality, self-mutilation, drug
    addiction, and alcoholism than other patients. Perhaps the most impressive
    finding of studies employing a ``symptom check-list" approach is the sheer
    length of the list of symptoms found to be significantly related to a
    history of childhood abuse (Browne and Finkelhor, 1986). From this wide
    array of symptoms, I have selected three categories that do not readily
    fall within the classic diagnostic criteria for PTSD: these are the
    somatic, dissociative, and affective sequelae of prolonged trauma.

    成為患者的倖存者會出現大量和各種各樣的抱怨。 他們的一般痛苦程度高於沒有
    虐待史的患者。 他們的症狀的詳細清單揭示了多個領域的重要病理:軀體、認知、
    情感、行為和關係。 Bryer 和他的同事 (1987) 在研究精神病住院患者時報告說,
    有身體虐待或性虐待史的女性在軀體化、抑鬱、一般和恐懼性焦慮、人際關係
    敏感性、偏執狂和“精神病”等標準化指標上的得分明顯高於其他患者 ”
    (未具體測量解離症狀)。 Briere (1988) 研究了危機干預服務的門診病人,
    報告說童年虐待的倖存者比其他病人表現出明顯更多的失眠、性功能障礙、分離、
    憤怒、自殺、自殘、吸毒和酗酒。 也許採用“症狀清單”方法的研究中最令人
    印象深刻的發現是發現與童年虐待史顯著相關的症狀清單的絕對長度(Browne 和
    Finkelhor,1986)。 從這些廣泛的症狀中,我選擇了三類不容易屬於 PTSD 經典
    診斷標準的類別:它們是長期創傷的軀體、解離和情感後遺症。

\paragraph{Somatization 軀體化}
    Repetitive trauma appears to amplify and generalize the physiologic
    symptoms of PTSD. Chronically traumatized people are hypervigilant, anxious
    and agitated, without any recognizable baseline state of calm or comfort
    (Hilberman, 1980). Over time, they begin to complain, not only of insomnia,
    startle reactions and agitation, but also of numerous other somatic
    symptoms. Tension headaches, gastrointestinal disturbances, and abdominal,
    back, or pelvic pain are extremely common. Survivors also frequently
    complain of tremors, choking sensations, or nausea. In clinical studies of
    survivors of the Nazi Holocaust, psychosomatic reactions were found to be
    practically universal (Hoppe, 1968; Krystal and Niederland, 1968; De Loos,
    1990). Similar observations are now reported in refugees from the
    concentration camps of Southeast Asia (Kroll et aL, 1989; Kinzie et aL,
    1990). Some survivors may conceptualize the damage of their prolonged
    captivity primarily in somatic terms. Nonspecific somatic symptoms appear
    to be extremely durable and may in fact increase over time (van der Ploerd,
    1989).

    重複性創傷似乎會放大和普遍化 PTSD 的生理症狀。 長期受創傷的人高度警惕、焦慮
    和激動,沒有任何可識別的平靜或舒適基線狀態(Hilberman,1980)。 隨著時間的
    推移,他們開始抱怨,不僅是失眠、驚嚇反應和激動,還有許多其他軀體症狀。
    緊張性頭痛、胃腸道不適以及腹部、背部或骨盆疼痛極為常見。 倖存者還經常抱怨
    震顫、窒息感或噁心。 在對納粹大屠殺倖存者的臨床研究中,發現身心反應實際上
    是普遍存在的(Hoppe,1968 年;Krystal 和 Niederland,1968 年;De Loos,
    1990 年)。 現在報導了來自東南亞集中營的難民的類似觀察結果(Kroll 等人,
    1989 年;Kinzie 等人,1990 年)。 一些倖存者可能主要從軀體的角度來概念化
    他們長期被囚禁的傷害。 非特異性軀體症狀似乎非常持久,實際上可能會隨著時間的
    推移而增加(van der Ploerd,1989)。

    The clinical literature also suggests an association between somatization
    disorders and childhood trauma, Briquet's initial descriptions of the
    disorder which now bears his name are filled with anecdotal references to
    domestic violence and child abuse. In a study of 87 children under twelve
    with hysteria, Briquet noted that one-third had been ``habitually
    mistreated or held constantly in fear or had been directed harshly by their
    parents." In another ten percent, he attributed the children's symptoms to
    traumatic experiences other than parental abuse (Mai and Merskey, 1980). A
    recent controlled study of 60 women with somatization disorder (Morrison,
    1989) found that 55\% had been sexually molested in childhood, usually by
    relatives. The study focused only on early sexual experiences; patients
    were not asked about physical abuse or about the more general climate of
    violence in their families. Systematic investigation of the childhood
    histories of patients with somatization disorder has yet to be undertaken.

    臨床文獻還表明軀體化障礙與童年創傷之間存在關聯,Briquet 對現在以他的名字
    命名的疾病的最初描述充滿了家庭暴力和虐待兒童的軼事。 在一項針對 87 名 12 歲
    以下歇斯底里症兒童的研究中,Briquet 指出,三分之一的人“經常受到虐待或一直
    處於恐懼之中,或者受到父母的嚴厲指揮”。 在另外百分之十的情況下,他將孩子的
    症狀歸因於父母虐待以外的創傷經歷(Mai 和 Merskey,1980)。 最近一項對 60 名
    患有軀體化障礙的女性進行的對照研究(Morrison,1989 年)發現,55\% 的女性
    在童年時期曾遭受過性騷擾,通常是親屬。 該研究僅關注早期的性經歷; 患者沒有
    被問及身體虐待或家庭中更普遍的暴力氣氛。 尚未對軀體化障礙患者的童年史進行
    系統調查。

\paragraph{Dissociation 解離}
    People in captivity become adept practitioners of the arts of altered
    consciousness. Through the practice of dissociation, voluntary thought
    suppression, minimization, and sometimes outright denial, they learn to
    alter an unbearable reality. Prisoners frequently instruct one another in
    the induction of trance states. These methods are consciously applied to
    withstand hunger, cold, and pain (Partnoy, 1986; Sharansky, 1988). During
    prolonged confinement and isolation, some prisoners are able to develop
    trance capabilities ordinarily seen only in extremely hypnotizable people,
    including the ability to form positive and negative hallucinations, and to
    dissociate parts of the personality. (See first-person accounts by Elaine
    Mohamed in Russell (1989) and by Mauricio Rosencof in Weschler (1989).)
    Disturbances in time sense, memory, and concentration are almost
    universally reported (Allodi, 1985; Tennant et aL, 1986; Kinzie et aL,
    1984). Alterations in time sense begin with the obliteration of the future
    but eventually progress to the obliteration of the past (Levi, 1958). The
    rupture in continuity between present and past frequently persists even
    after the prisoner is released. The prisoner may give the appearance of
    returning to ordinary time, while psychologically remaining bound in the
    timelessness of the prison (Jaffe, 1968).

    被囚禁的人成為改變意識藝術的熟練實踐者。 通過解離、自願抑制思想、最小化,
    有時甚至是徹底否認,他們學會了改變無法忍受的現實。 囚犯經常互相指導如何進入
    恍惚狀態。 這些方法被有意識地用於抵禦飢餓、寒冷和痛苦(Partnoy,1986;
    Sharansky,1988)。 在長期的監禁和隔離期間,一些囚犯能夠發展出通常只有極易
    被催眠的人才能看到的催眠能力,包括形成積極和消極幻覺的能力,以及分離部分人
    格的能力。 (參見 Russell (1989) 中 Elaine Mohamed 和 Weschler (1989) 中
    Mauricio Rosencof 的第一人稱描述。)時間感、記憶和注意力方面的障礙幾乎被
    普遍報導(Allodi,1985 年;Tennant 等人,1986 年) ;Kinzie 等人,1984 年)。
    時間感的改變始於對未來的抹殺,但最終發展為對過去的抹殺(Levi,1958)。 即使在
    囚犯獲釋後,現在和過去之間的連續性斷裂也經常存在。 囚犯可能會表現出回到
    平常時間的樣子,但在心理上仍然被監獄的永恆所束縛 (Jaffe, 1968)。

    In survivors of prolonged childhood abuse, these dissociative capacities
    are developed to the extreme. Shengold (1989) describes the
    ``mind-fragmenting operations" elaborated by abused children in order to
    preserve ``the delusion of good parents." He notes the ``establishment of
    isolated divisions of the mind in which contradictory images of the self
    and of the parents are never permitted to coalesce." The virtuosic feats of
    dissociation seen, for example, in multiple personality disorder, are
    almost always associated with a childhood history of massive and prolonged
    abuse (Putnam et aL, 1986; Putnam, 1989; Ross et at, 1990). A similar
    association between severity of childhood abuse and extent of dissociative
    symptomatology has been documented in subjects with borderline personality
    disorder (Herman et aL, 1989), and in a nonclinical, college-student
    population (Sanders et al„ 1989).

    在長期童年虐待的倖存者中,這些分離能力發展到了極致。 Shengold (1989) 描述了
    受虐兒童為了保持“好父母的錯覺”而精心設計的“精神分裂行動”。 他注意到“思想的
    孤立分裂的建立,在這種分裂中,自我和父母的相互矛盾的形象永遠不允許合併。”
    例如,在多重人格障礙中看到的精湛的解離技藝幾乎總是與童年時期遭受大規模和
    長期虐待的歷史有關(Putnam 等人,1986 年;Putnam,1989 年;Ross 等人,
    1990 年)。 在患有邊緣性人格障礙的受試者(Herman 等人,1989 年)和非臨床
    大學生群體(Sanders 等人,1989 年)中,已經記錄了童年虐待的嚴重程度與
    解離症狀的程度之間的類似關聯。

    There are people with very strong and secure belief systems, who can endure
    the ordeals of prolonged abuse and emerge with their faith intact. But
    these are the extraordinary few. The majority experience the bitterness of
    being forsaken by man and God (Wiesel, 1960). These staggering
    psychological losses most commonly result in a tenacious state of
    depression. Protracted depression is reported as the most common finding in
    virtually all clinical studies of chronically traumatized people
    (Goldstein et al., 1987) Herman, 1981; Hilberman, 1980; Kinzie et al.,
    1984; Krystal, 1968; Walker, 1979). Every aspect of the experience of
    prolonged trauma combines to aggravate depressive symptoms. The chronic
    hyperarousal and intrusive symptoms of PTSD fuse with the vegetative
    symptoms of depression, producing what Niederland calls the ``survivor
    triad" of insomnia, nightmares, and psychosomatic complaints (in Krystal,
    1968, p. 313). The dissociative symptoms of PTSD merge with the
    concentration difficulties of depression. The paralysis of initiative of
    chronic trauma combines with the apathy and helplessness of depression. The
    disruptions in attachments of chronic trauma reinforce the isolation and
    withdrawal of depression. The debased self image of chronic trauma fuels
    the guilty ruminations of depression. And the loss of faith suffered in
    chronic trauma merges with the hopelessness of depression.

    有些人擁有非常強大和安全的信仰體系,他們可以忍受長期虐待的磨難,並以
    完好無損的信仰出現。 但這些是極少數。 大多數人都經歷過被人和上帝拋棄的痛苦
    (Wiesel, 1960)。 這些驚人的心理損失通常會導致頑固的抑鬱狀態。 據報導,
    長期抑鬱是幾乎所有慢性創傷患者臨床研究中最常見的發現
    (Goldstein 等,1987)Herman,1981; 希爾伯曼,1980; Kinzie 等人,1984 年;
    克里斯托,1968 年; 沃克,1979)。 長期創傷經歷的各個方面都會加重抑鬱症狀。
    PTSD 的慢性過度興奮和侵入性症狀與抑鬱症的植物性症狀融合在一起,產生了
    Niederland 所說的失眠、噩夢和心身不適的“倖存者三聯徵”(Krystal,1968 年,
    第 313 頁)。 PTSD 的解離症狀與抑鬱症的注意力難以集中在一起。 慢性創傷的
    主動性癱瘓與抑鬱症的冷漠和無助相結合。 慢性創傷對依戀的破壞加強了抑鬱症的
    孤立和退縮。 慢性創傷的貶低自我形象助長了抑鬱症的內疚反思。 在慢性創傷中
    遭受的信仰喪失與抑鬱症的絕望融合在一起。

    The humiliated rage of the imprisoned person also adds to the depressive
    burden (Hilberman, 1980). During captivity, the prisoner can not express
    anger at the perpetrator; to do so would jepordize survival. Even after
    release, the survivor may continue to fear retribution for any expression
    of anger against the captor. Moreover, the survivor carries a burden of
    unexpressed anger against all those who remained indifferent and failed to
    help. Efforts to control this rage may further exacerbate the survivor's
    social withdrawal and paralysis of initiative. Occasional outbursts of rage
    against others may further alienate the survivor and prevent the
    restoration of relationships. And internalization of rage may result in a
    malignant self-hatred and chronic sucidality. Epidemiologic studies of
    returned POWs consistently document increased mortality as the result of
    homicide, suicide, and suspicious accidents (Segal et al., 1976). Studies
    of battered women similarly report a tenacious suicidality. In one clinical
    series of 100 battered women, 42\% had attempted suicide (Gayford, 1975).
    While major depression is frequently diagnosed in survivors of prolonged
    abuse, the connection with the trauma is frequently lost. Patients are
    incompletely treated when the traumatic origins of the intractable
    depression are not recognized (Kinzie et al., 1990).

    被監禁者的屈辱憤怒也增加了抑鬱的負擔 (Hilberman, 1980)。 在囚禁期間,囚犯
    不能對肇事者表達憤怒; 這樣做會危及生存。 即使在獲釋後,倖存者仍可能繼續
    害怕因對綁架者表達憤怒而遭到報復。 此外,倖存者對所有漠不關心、未能提供
    幫助的人懷有無法表達的憤怒。 控制這種憤怒的努力可能會進一步加劇倖存者的
    社交退縮和主動性癱瘓。 偶爾對他人大發雷霆可能會進一步疏遠倖存者並阻礙關係的
    恢復。 憤怒的內化可能導致惡性的自我仇恨和慢性自殺。 對返回的戰俘進行的
    流行病學研究一致證明,殺人、自殺和可疑事故導致死亡率增加(Segal 等人,
    1976 年)。 對受虐婦女的研究同樣報告了頑固的自殺傾向。 在一項包含 100 名
    受虐婦女的臨床系列研究中,42\% 的人曾試圖自殺(Gayford,1975 年)。 雖然
    長期虐待的倖存者經常被診斷出重度抑鬱症,但與創傷的聯繫卻經常消失。 當無法
    識別頑固性抑鬱症的創傷性起源時,患者就無法得到完全治療(Kinzie 等人,
    1990 年)。

\section{Characterological Sequelae of Prolonged Victimization
長期受害的性格後遺症}

\paragraph{Pathological Changes in Relationship 關係的病理變化}
    In situations of captivity, the perpetrator becomes the most powerful
    person in the life of the victim, and the psychology of victim is shaped
    over time by the actions and beliefs of the perpetrator. The methods which
    enable one human being to control another are remarkably consistent. These
    methods were first systematically detailed in reports of so-called
    ``brainwashing" in American prisoners of war (Biderman, 1957; Farber et aL,
    1957). Subsequently, Amnesty International (1973) published a systematic
    review of methods of coercion, drawing upon the testimony of political
    prisoners from widely differing cultures. The accounts of coercive methods
    given by battered women (Dobash and Dobash, 1979; NiCarthy, 1982, Walker,
    1979), abused children (Rhodes, 1990), and coerced prostitutes (Lovelace
    and McGrady, 1980) bear an uncanny resemblance to those hostages, political
    prisoners, and survivors of concentration camps. While perpetrators of
    organized political or sexual exploitation may instruct each other in
    coercive methods, perpetrators of domestic abuse appear to reinvent them.

    在囚禁的情況下,犯罪者成為受害者生命中最有權勢的人,受害者的心理隨著時間的
    推移被犯罪者的行為和信念所塑造。 使一個人能夠控制另一個人的方法非常一致。
    這些方法首先在美國戰俘所謂的“洗腦”報告中得到系統詳述(Biderman,1957 年;
    Farber 等人,1957 年)。 隨後,國際特赦組織 (Amnesty International, 1973)
    發表了對脅迫方法的系統回顧,借鑒了來自不同文化背景的政治犯的證詞。 受虐婦女
    (Dobash and Dobash, 1979; NiCarthy, 1982, Walker, 1979)、受虐兒童 (Rhodes,
    1990) 和被脅迫妓女 (Lovelace and McGrady, 1980) 那些人質、政治犯和集中營的
    倖存者。 雖然有組織的政治剝削或性剝削的肇事者可能會以強制方法相互指導,但
    家庭暴力的肇事者似乎會重新發明它們。

    The methods of establishing control over another person are based upon the
    systematic, repetitive infliction of psychological trauma. These methods
    are designed to instill terror and helplessness, to destroy the victim's
    sense of self in relation to others, and to foster a pathologic attachment
    to the perpetrator. Although violence is a universal method of instilling
    terror, the threat of death or serious harm, either to the victim or to
    others close to her, is much more frequent than the actual resort to
    violence. Fear is also increased by unpredictable outbursts of violence,
    and by inconsistent enforcement of numerous trivial demands and petty
    rules.

    建立對另一個人的控制的方法是基於系統的、重複的心理創傷。 這些方法旨在灌輸
    恐懼和無助,破壞受害者與他人的自我意識,並培養對肇事者的病態依戀。 儘管暴力
    是一種普遍的灌輸恐怖的方法,但對受害者或與她關係密切的其他人造成死亡或嚴重
    傷害的威脅比實際訴諸暴力要頻繁得多。 不可預測的暴力爆發,以及對許多瑣碎要求
    和細則的不一致執行,也會增加恐懼。

    In addition to inducing terror, the perpetrator seeks to destroy the
    victim's sense of autonomy. This is achieved by control of the victim's
    body and bodily functions. Deprivation of food, sleep, shelter, exercise,
    personal hygiene, or privacy are common practices. Once the perpetrator has
    established this degree of control, he becomes a potential source of solace
    as well as humiliation. The capricious granting of small indulgences may
    undermine the psychological resistance of the victim far more effectively
    than unremitting deprivation and fear.

    除了誘發恐懼之外,肇事者還試圖破壞受害者的自主意識。 這是通過控制受害者的
    身體和身體機能來實現的。 剝奪食物、睡眠、住所、鍛煉、個人衛生或隱私是常見的
    做法。 一旦肇事者建立了這種程度的控制,他就會成為安慰和羞辱的潛在來源。 
    反覆無常地給予小額寬恕可能比持續不斷的剝奪和恐懼更有效地削弱受害者的心理
    抵抗力。

    As long as the victim maintains strong relationships with others, the
    perpetrator's power is limited; invariably, therefore, he seeks to isolate
    his victim. The perpetrator will not only attempt to prohibit communication
    and material support, but will also try to destroy the victim's emotional
    ties to others. The final step in the ``breaking" of the victim is not
    completed until she has been forced to betray her most basic attachments,
    by witnessing or participating in crimes against others.

    只要受害者與他人保持牢固的關係,加害者的權力就有限; 因此,他總是試圖孤立
    他的受害者。 肇事者不僅會試圖禁止交流和物質支持,還會試圖破壞受害者與他人的
    情感聯繫。 直到受害者通過目睹或參與對他人的犯罪而被迫背叛她最基本的依戀時,
    受害者“崩潰”的最後一步才算完成。

    As the victim is isolated, she becomes increasingly dependent upon the
    perpetrator, not only for survival and basic bodily needs, but also for
    information and even for emotional sustenance. Prolonged confinement in
    fear of death and in isolation reliably produces a bond of identification
    between captor and victim. This is the ``traumatic bonding" that occurs in
    hostages, who come to view their captors as their saviors and to fear and
    hate their rescuers. Symonds (1982) describes this process as an enforced
    regression to ``psychological infantilism" which ``compels victims to cling
    to the very person who is endangering their life." The same traumatic
    bonding may occur between a battered woman and her abuser (Dutton and
    Painter, 1981; Graham et aL, 1988), or between an abused child and abusive
    parent (Herman, 1981; van der Kolk, 1987). Similar experiences are also
    reported by people who have been inducted into totalitarian religious cults
    (Halperin, 1983; Lifton, 1987).

    由於受害者是孤立的,她變得越來越依賴肇事者,不僅是為了生存和基本的身體
    需求,也是為了獲得信息,甚至是為了情感寄託。 因害怕死亡和隔離而長期監禁
    確實會在綁架者和受害者之間產生一種認同感。 這是發生在人質身上的
    “創傷性聯結”,他們開始將綁架者視為自己的救世主,並害怕和憎恨拯救他們的人。
    西蒙茲 (Symonds, 1982) 將這一過程描述為“心理幼稚主義”的強制倒退,
    “迫使受害者緊緊抓住危及他們生命的人”。 同樣的創傷性聯繫可能發生在受虐婦女
    和施虐者之間(Dutton 和 Painter,1981;Graham 等人,1988),或者受虐兒童和
    施虐父母之間(Herman,1981;van der Kolk,1987)。 被引導到極權主義宗教
    邪教的人也報告了類似的經歷 (Halperin, 1983; Lifton, 1987)。

    With increased dependency upon the perpetrator comes a constriction in
    initiative and planning. Prisoners who have not been entirely ``broken" do
    not give up the capacity for active engagement with their environment. On
    the contrary, they often approach the small daily tasks of survival with
    extraordinary ingenuity and determination. But the field of initiative is
    increasingly narrowed within confines dictated by the perpetrator. The
    prisoner no longer thinks of how to escape, but rather of how to stay
    alive, or how to make captivity more bearable. This narrowing in the range
    of initiative becomes habitual with prolonged captivity, and must be
    unlearned after the prisoner is liberated. (See, for example, the testimony
    of Hearst (1982) and Rosencof in Weschler, 1989.)

    隨著對肇事者的依賴程度的增加,主動性和計劃性受到限制。 沒有完全“崩潰”的囚犯
    不會放棄積極參與環境的能力。 相反,他們常常以非凡的聰明才智和決心來處理
    日常的小生存任務。 但是,在肇事者指定的範圍內,主動權的範圍越來越窄。 囚犯
    不再想著如何逃脫,而是想著如何活下去,或者如何讓囚禁更能忍受。 這種主動
    範圍的縮小隨著長時間的囚禁而成為習慣,並且在囚犯被釋放後必須忘卻。
    (例如,參見 Hearst (1982) 和 Rosencof 在 Weschler, 1989 中的證詞。)

    Because of this constriction in the capacities for active engagement with
    the world, chronically traumatized people are often described as passive or
    helpless. Some theorists have in fact applied the concept of ``learned
    helplessness" to the situation of battered women and other chronically
    traumatized people (Walker, 1979; van der Kolk, 1987). Prolonged captivity
    undermines or destroys the ordinary sense of a relatively safe sphere of
    initiative, in which there is some tolerance for trial and error. To the
    chronically traumatized person, any independent action is insubordination,
    which carries the risk of dire punishment.

    由於這種積極參與世界的能力受到限制,長期受創傷的人通常被描述為被動或無助。
    事實上,一些理論家已經將“習得性無助”的概念應用於受虐婦女和其他長期遭受
    創傷的人的處境(Walker,1979;van der Kolk,1987)。 長時間的囚禁破壞或
    破壞了相對安全的主動領域的普通感覺,在這種領域中,對試錯有一定的容忍度。
    對於長期受創傷的人來說,任何獨立的行動都是不服從命令,有遭受可怕懲罰的
    風險。

    The sense that the perpetrator is still present, even after liberation,
    signifies a major alteration in the survivor's relational world. The
    enforced relationship, which of necessity monopolizes the victim's
    attention during captivity, becomes part of her inner life and continues to
    engross her attention after release. In political prisoners, this continued
    relationship may take the form of a brooding preoccupation with the
    criminal careers of specific perpetrators or with more abstract concerns
    about the unchecked forces of evil in the world. Released prisoners
    continue to track their captors, and to fear them (Krystal, 1968). In
    sexual, domestic, and religious cult prisoners, this continued relationship
    may take a more ambivalent form: the survivor may continue to fear her
    former captor, and to expect that he will eventually hunt her down; she may
    also feel empty, confused, and worthless without him (Walker, 1979).

    即使在解放之後,肇事者仍然存在的感覺表明倖存者的關係世界發生了重大變化。
    這種強迫關係必然會在囚禁期間壟斷受害者的注意力,成為她內心生活的一部分,
    並在獲釋後繼續佔據她的注意力。 在政治犯身上,這種持續的關係可能表現為對
    特定犯罪者的犯罪生涯的沉思,或者對世界上不受約束的邪惡勢力的更抽象的關注。
    被釋放的囚犯繼續追踪他們的俘虜,並害怕他們 (Krystal, 1968)。 在性、家庭和
    宗教邪教囚犯中,這種持續的關係可能採取更加矛盾的形式:倖存者可能繼續害怕
    她的前俘虜,並期望他最終會追捕她; 沒有他,她也可能會感到空虛、困惑和
    一文不值 (Walker, 1979)。

    Even after escape, it is not possible simply to reconstitute relationships
    of the sort that existed prior to captivity. All relationships are now
    viewed through the lens of extremity. Just as there is no range of moderate
    engagement or risk for initiative, there is no range of moderate engagement
    or risk for relationship. The survivor approaches all relationships as
    though questions of life and death are at stake, oscillating between
    intense attachment and terrified withdrawal.

    即使在逃脫之後,也不可能簡單地重建囚禁前存在的那種關係。 現在,所有的關係
    都是通過極端的鏡頭來看待的。 正如主動性沒有適度參與或風險的範圍一樣,
    關係也沒有適度參與或風險的範圍。 倖存者對待所有的關係,就好像生死攸關的
    問題,在強烈的依戀和恐懼的退縮之間搖擺不定。

    In survivors of childhood abuse, these disturbances in relationship are
    further amplified. Oscillations in attachment, with formation of intense,
    unstable relationships, are frequently observed. These disturbances are
    described most fully in patients with borderline personality disorder, the
    majority of whom have extensive histories of childhood abuse. A recent
    empirical study, confirming a vast literature of clinical observations,
    outlines in detail the specific pattern of relational difficulties. Such
    patients find it very hard to tolerate being alone, but are also
    exceedingly wary of others. Terrified of abandonment on the one hand, and
    domination on the other, they oscillate between extremes of abject
    submissiveness and furious rebellion (Melges and Swartz, 1989). They tend
    to form ``special" dependent relations with idealized caretakers in which
    ordinary boundaries are not observed (Zanarini et aL, 1990). Very similar
    patterns are described in patients with MPD, including the tendency to
    develop intense, highly ``special" relationships ridden with boundary
    violations, conflict, and potential for exploitation (Kluft, 1990).

    在童年虐待的倖存者中,這些關係中的障礙進一步放大。 經常觀察到依戀的波動,
    伴隨著強烈的、不穩定的關係的形成。 這些障礙在邊緣型人格障礙患者中得到了
    最充分的描述,他們中的大多數都有廣泛的童年虐待史。 最近的一項實證研究證實了
    大量的臨床觀察文獻,詳細概述了關係困難的具體模式。 這些患者很難忍受獨處,
    但對他人也極度警惕。 一方面害怕被遺棄,另一方面害怕統治,他們在卑鄙的順從
    和憤怒的反叛這兩個極端之間搖擺不定(Melges 和 Swartz,1989)。 他們傾向於
    與理想化的照顧者形成“特殊”的依賴關係,在這種關係中,通常的界限是不被遵守的
    (Zanarini 等人,1990)。 在 MPD 患者中描述了非常相似的模式,包括傾向於
    發展強烈的、高度“特殊”的關係,充滿邊界侵犯、衝突和潛在的剝削 (Kluft,
    1990)。

\paragraph{Pathologic Changes in Identity 身份的病理變化}
    Subjection to a relationship of coercive control produces profound
    alterations in the victim's identity. All the structures of the self -- the
    image of the body, the internalized images of others, and the values and
    ideals that lend a sense of coherence and purpose -- are invaded and
    systematically broken down. In some totalitarian systems (political,
    religious, or sexual/domestic), this process reaches the extent of taking
    away the victim's name (Hearst and Moscow, 1982; Lovelace and McGrady).
    While the victim of a single acute trauma may say she is ``not herself"
    since the event, the victim of chronic trauma may lose the sense that she
    has a self. Survivors may describe themselves as reduced to a nonhuman life
    form (Lovelace and McGrady, 1980; Timerman, 1981). Niederland (1968), in
    his clinical observations of concentration camp survivors, noted that
    alterations of personal identity were a constant feature of the survivor
    syndrome. While the majority of his patients complained, ``I am now a
    different person," the most severely harmed stated simply, ``I am not a
    person."

    屈從於強制控制關係會使受害者的身份發生深刻的變化。 自我的所有結構——身體的
    形象、他人的內化形象,以及賦予連貫性和目的感的價值觀和理想——都被入侵並被
    系統地分解。 在一些極權制度(政治、宗教或性/家庭)中,這個過程達到了剝奪
    受害者姓名的程度(Hearst 和 Moscow,1982;Lovelace 和 McGrady)。 一次急性
    創傷的受害者可能會說自事件發生後她“不是她自己”,而慢性創傷的受害者可能會
    失去她擁有自我的感覺。 倖存者可能會將自己描述為一種非人類生命形式(Lovelace
    和 McGrady,1980 年;Timerman,1981 年)。 Niederland (1968) 在他對集中營
    倖存者的臨床觀察中指出,個人身份的改變是倖存者綜合症的一個不變特徵。 
    雖然他的大多數患者抱怨說,“我現在是另一個人”,但受傷害最嚴重的人只是
    簡單地說,“我不是一個人。”

    Survivors of childhood abuse develop even more complex deformations of
    identity. A malignant sense of the self as contaminated, guilty, and evil
    is widely observed. Fragmentation in the sense of self is also common,
    reaching its most dramatic extreme in multiple personality disorder.
    Ferenczi (1933) describes the ``atomization" of the abused child's
    personality. Rieker and Carmen (1986) describe the central pathology in
    victimized children as a ``disordered and fragmented identity deriving from
    accommodations to the judgments of others." Disturbances in identity
    formation are also characteristic of patients with borderline and multiple
    personality disorders, the majority of whom have childhood histories of
    severe trauma. In MPD, the fragmentation of the self into dissociated
    alters is, of course, the central feature of the disorder (Bliss, 1986;
    Putnam, 1989). Patients with BPD, though they lack the dissociative
    capacity to form fragmented alters, have similar difficulties in the
    formation of an integrated identity. An unstable sense of self is
    recognized as one of the major diagnostic criteria for BPD, and the
    ``splitting" of inner representations of self and others is considered by
    some theorists to be the central underlying pathology of the disorder
    (Kernberg, 1967).

    童年虐待的倖存者會發展出更複雜的身份變形。 人們普遍觀察到一種自我被污染、
    有罪和邪惡的惡性感覺。 自我意識的分裂也很常見,在多重人格障礙中達到最
    戲劇性的極端。 Ferenczi (1933) 描述了受虐兒童人格的“霧化”。 Rieker 和
    Carmen (1986) 將受害兒童的核心病理學描述為“因適應他人的判斷而產生的混亂
    和支離破碎的身份認同”。 身份形成障礙也是邊緣型人格障礙和多重人格障礙患者的
    特徵,其中大多數人都有嚴重創傷的童年史。 在 MPD 中,自我分裂成分離的分身
    當然是該障礙的核心特徵 (Bliss, 1986; Putnam, 1989)。 BPD 患者雖然缺乏形成
    碎片化人格的解離能力,但在形成完整身份方面也有類似的困難。 不穩定的自我意識
    被認為是 BPD 的主要診斷標準之一,一些理論家認為自我和他人的內在表徵的
    “分裂”是該障礙的核心潛在病理學 (Kernberg, 1967)。

\paragraph{Repetition of Harm Following Prolonged Victimization
長期受害後傷害的重複}
    Repetitive phenomena have been widely noted to be sequelae of severe
    trauma. The topic has been recently reviewed in depth by van der Kolk
    (1989). In simple PTSD, these repetitive phenomena may take the form of
    intrusive memories, somato-sensory reliving experiences, or behavioral
    re-enactments of the trauma (Brett and Ostroff, 1985; Terr, 1983). After
    prolonged and repeated trauma, by contrast, survivors may be at risk for
    repeated harm, either self-inflicted, or at the hands of others. These
    repetitive phenomena do not bear a direct relation to the original trauma;
    they are not simple reenactments or reliving experiences. Rather, they take
    a disguised symptomatic or characterological form.

    重複現像已被廣泛認為是嚴重創傷的後遺症。 van der Kolk (1989) 最近對這個話題
    進行了深入的回顧。 在簡單的 PTSD 中,這些重複現象可能會以侵入性記憶、
    軀體感覺重溫體驗或創傷的行為重演的形式出現(Brett 和 Ostroff,1985 年;
    Terr,1983 年)。 相比之下,在長期和反覆的創傷之後,倖存者可能面臨
    反覆傷害的風險,無論是自己造成的,還是他人的手。 這些重複的現象與最初的
    創傷沒有直接關係; 它們不是簡單的重演或重溫體驗。 相反,它們採取了一種
    變相的症狀或特徵形式。

    About 7-10\% of psychiatric patients are thought to injure themselves
    deliberately (Favazza and Conterio, 1988). Self-mutilization is a
    repetitive behavior which appears to be quite distinct from attempted
    suicide. This compulsive form of self-injury appears to be strongly
    associated with a history of prolonged repeated trauma. Self-mutilation,
    which is rarely seen after a single acute trauma, is a common sequel of
    protracted childhood abuse (Briere, 1988; van der Kolk et al., 1991).
    Self-injury and other paroxysmal forms of attack on the body have been
    shown to develop most commonly in those victims whose abuse began early in
    childhood (van der Kolk, 1992).

    據認為,大約 7-10\% 的精神病患者會故意傷害自己(Favazza 和 Conterio,
    1988 年)。 自殘是一種重複行為,似乎與自殺未遂截然不同。 這種強迫性自傷形式
    似乎與長期反覆創傷的歷史密切相關。 自殘在一次急性創傷後很少見,是長期童年
    虐待的常見後果(Briere,1988 年;van der Kolk 等人,1991 年)。 自殘和其他
    陣發性身體攻擊已被證明最常發生在童年早期就開始受虐待的受害者身上(van der
    Kolk,1992 年)。

    The phenomenon of repeated victimization also appears to be specifically
    associated with histories of prolonged childhood abuse. Widescale
    epidemiologic studies provide strong evidence that survivors of childhood
    abuse are at increased risk for repeated harm in adult life. For example,
    the risk of rape, sexual harassment, and battering, though very high for
    all women, is approximately doubled for survivors of childhood sexual abuse
    (Russell, 1986). One clinical observer goes so far as to label this
    phenomenon the ``sitting duck syndrome" (Kluft, 1990).

    反覆受害的現像似乎也與長期的童年虐待史有關。 廣泛的流行病學研究提供了
    強有力的證據,表明童年虐待的倖存者在成年後遭受反覆傷害的風險增加。 例如,
    強姦、性騷擾和毆打的風險,雖然對所有女性來說都很高,但對於兒童期
    性虐待的倖存者來說,風險大約增加一倍 (Russell, 1986)。 一位臨床觀察員甚至
    將這種現象稱為“坐鴨綜合症”(Kluft,1990)。

    In the most extreme cases, survivors of childhood abuse may find themselves
    involved in abuse of others, either in the role of passive bystander or,
    more rarely, as a perpetrator. Burgess and her collaborators (1984), for
    example, report that children who had been exploited in a sex ring for more
    than one year were likely to adopt the belief system of the perpetrator and
    to become exploitative toward others. A history of prolonged childhood
    abuse does appear to be a risk factor for becoming an abuser, especially in
    men (Herman, 1988; Hotaling and Sugarman, 1986). In women, a history of
    witnessing domestic violence (Hotaling and Sugarman, 1986), or sexual
    victimization (Goodwin et aL, 1982) in childhood appears to increase the
    risk of subsequent marriage to an abusive mate. It should be noted,
    however, that contrary to the popular notion of a ``generational cycle of
    abuse," the great majority of survivors do not abuse others (Kaufman and
    Zigler, 1987). For the sake of their children, survivors frequently
    mobilize caring and protective capacities that they have never been able to
    extend to themselves (Coons, 1985).

    在最極端的情況下,童年虐待的倖存者可能會發現自己捲入了虐待他人的行列,要么
    扮演被動旁觀者的角色,要么更罕見地成為施虐者。 例如,Burgess 和她的合作者
    (1984) 報告說,在性環中被剝削超過一年的兒童很可能會接受犯罪者的信仰體系,
    並對他人產生剝削。 長期的童年虐待史似乎確實是成為施虐者的一個危險因素,
    尤其是對男性而言(Herman,1988 年;Hotaling 和 Sugarman,1986 年)。
    在女性中,童年時期目睹家庭暴力(Hotaling 和 Sugarman,1986 年)或性受害
    (Goodwin 等人,1982 年)的歷史似乎會增加隨後與虐待配偶結婚的風險。 然而,
    應該指出的是,與流行的“虐待代際循環”概念相反,絕大多數倖存者並不虐待他人
    (Kaufman 和 Zigler,1987)。 為了他們的孩子,倖存者經常動員他們從未能夠
    延伸到自己身上的關懷和保護能力(Coons,1985)。

\section{Conclusions 結論}

    The review of the literature offers unsystematized but extensive empirical
    support for the concept of a complex post-traumatic syndrome in survivors
    of prolonged, repeated victimization. This previously undefined syndrome
    may coexist with simple PTSD, but extends beyond it. The syndrome is
    characterized by a pleomorphic symptom picture, enduring personality
    changes, and high risk for repeated harm, either selFinflicted or at the
    hands of others.
    文獻回顧為長期、反覆受害的倖存者的複雜創傷後綜合症的概念提供了
    非系統但廣泛的實證支持。 這種以前未定義的綜合症可能與簡單的 PTSD 共存,
    但超出了它。 該綜合徵的特點是多形性症狀、持久的性格變化以及反覆傷害的
    高風險,無論是自己造成的還是他人造成的。

    Failure to recognize this syndrome as a predictable consequence of
    prolonged, repeated trauma contributes to the misunderstanding of
    survivors, a misunderstanding shared by the general society and the mental
    health professions alike. Social judgment of chronically traumatized people
    has tended to be harsh (Biderman and Zimmer, 1961; Wardell et aL, 1983).
    The propensity to fault the character of victims can be seen even in the
    case of politically organized mass murder. Thus, for example, the aftermath
    of the Nazi Holocaust witnessed a protracted intellectual debate regarding
    the ``passivity" of the Jews, and even their ``complicity" in their fate
    (Dawidowicz, 1975). Observers who have never experienced prolonged terror,
    and who have no understanding of coercive methods of control, often presume
    that they would show greater psychological resistance than the victim in
    similar circumstances. The survivor's difficulties are all too easily
    attributed to underlying character problems, even when the trauma is known.
    When the trauma is kept secret, as is frequently the case in sexual and
    domestic violence, the survivor's symptoms and behavior may appear quite
    baffling, not only to lay people but also to mental health professionals.

    未能認識到這種綜合症是長期、反覆創傷的可預測後果,這會導致對倖存者的誤解,
    這是整個社會和心理健康專業人士都存在的誤解。 對長期受創傷的人的社會判斷
    往往是嚴厲的(Biderman 和 Zimmer,1961;Wardell 等人,1983)。 即使在政治
    組織的大規模謀殺案中,也可以看出對受害者性格進行指責的傾向。 因此,例如,
    納粹大屠殺的後果見證了一場曠日持久的關於猶太人的“被動性”,甚至是他們在
    命運中的“同謀”的知識分子辯論(Dawidowicz,1975)。 從未經歷過長期恐怖並且
    不了解強制控制方法的觀察者通常認為,在類似情況下,他們會比受害者表現出
    更大的心理抵抗力。 倖存者的困難很容易歸因於潛在的性格問題,即使創傷
    是已知的。 當創傷被保密時,就像性暴力和家庭暴力中經常發生的情況一樣,
    倖存者的症狀和行為可能會顯得非常莫名其妙,不僅對於外行人,對於精神衛生
    專業人員也是如此。

    The clinical picture of a person who has been reduced to elemental concerns
    of survival is still frequently mistaken for a portrait of the survivor's
    underlying character. Concepts of personality developed in ordinary
    circumstances are frequently applied to survivors, without an understanding
    of the deformations of personality which occur under conditions of coercive
    control. Thus, patients who suffer from the complex sequelae of chronic
    trauma commonly risk being misdiagnosed as having personality disorders.
    They may be described as ``dependent," ``masochistic," or
    ``self-defeating." Earlier concepts of masochism or repetition compulsion
    might be more usefully supplanted by the concept of a complex traumatic
    syndrome.

    一個人的臨床表現已經淪為對生存的基本關注,但仍然經常被誤認為是倖存者
    潛在性格的寫照。 在一般情況下發展起來的人格概念經常被應用到倖存者身上,
    卻不了解在強制控制條件下發生的人格變形。 因此,患有慢性創傷複雜後遺症的患者
    通常有被誤診為人格障礙的風險。 他們可能被描述為“依賴”、“自虐”或“弄巧成拙”。
    受虐狂或重複強迫症的早期概念可能更有效地被複雜創傷綜合症的概念所取代。

    Misapplication of the concept of personality disorder may be the most
    stigmatizing diagnostic mistake, but it is by no means the only one. In
    general, the diagnostic concepts of the existing psychiatric canon,
    including simple PTSD, are not designed for survivors of prolonged,
    repeated trauma, and do not fit them well. The evidence reviewed in this
    paper offer strong support for expanding the concept of PTSD to include a
    spectrum of disorders (Brett, 1992), ranging from the brief, self-limited
    stress reaction to a single acute trauma, through simple PTSD, to the
    complex disorder of extreme stress (DESNOS) that follows upon prolonged
    exposure to repeated trauma.

    人格障礙概念的誤用可能是最具污名化的診斷錯誤,但絕不是唯一的錯誤。
    總的來說,現有精神病學經典的診斷概念,包括簡單的 PTSD,並不是為長期、反覆
    創傷的倖存者設計的,也不適合他們。 本文審查的證據有力地支持將 PTSD 的概念
    擴展到包括一系列障礙(Brett,1992),範圍從短暫的自限性應激反應到單一的
    急性創傷,從簡單的 PTSD 到復雜的 長期暴露於反覆創傷後出現的極度壓力障礙
    (DESNOS)。

\end{document}

